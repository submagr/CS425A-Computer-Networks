\documentclass[11pt,a4paper,titlepage]{article}
\usepackage[a4paper]{geometry}
\usepackage[utf8]{inputenc}
\usepackage[english]{babel}
\usepackage{lipsum}

\usepackage{amsmath, amssymb, amsfonts, amsthm, fouriernc, mathtools}
% mathtools for: Aboxed (put box on last equation in align envirenment)
\usepackage{microtype} %improves the spacing between words and letters

\usepackage{graphicx}
\graphicspath{ {./pics/} {./eps/}}
\usepackage{epsfig}
\usepackage{epstopdf}
\usepackage{listings}
%%%%%%%%%%%%%%%%%%%%%%%%%%%%%%%%%%%%%%%%%%%%%%%%%%
%% COLOR DEFINITIONS
%%%%%%%%%%%%%%%%%%%%%%%%%%%%%%%%%%%%%%%%%%%%%%%%%%
\usepackage[svgnames]{xcolor} % Enabling mixing colors and color's call by 'svgnames'
%%%%%%%%%%%%%%%%%%%%%%%%%%%%%%%%%%%%%%%%%%%%%%%%%%
\definecolor{MyColor1}{rgb}{0.2,0.4,0.6} %mix personal color
\newcommand{\textb}{\color{Black} \usefont{OT1}{lmss}{m}{n}}
\newcommand{\blue}{\color{MyColor1} \usefont{OT1}{lmss}{m}{n}}
\newcommand{\blueb}{\color{MyColor1} \usefont{OT1}{lmss}{b}{n}}
\newcommand{\red}{\color{LightCoral} \usefont{OT1}{lmss}{m}{n}}
\newcommand{\green}{\color{Turquoise} \usefont{OT1}{lmss}{m}{n}}
%%%%%%%%%%%%%%%%%%%%%%%%%%%%%%%%%%%%%%%%%%%%%%%%%%




%%%%%%%%%%%%%%%%%%%%%%%%%%%%%%%%%%%%%%%%%%%%%%%%%%
%% FONTS AND COLORS
%%%%%%%%%%%%%%%%%%%%%%%%%%%%%%%%%%%%%%%%%%%%%%%%%%
%    SECTIONS
%%%%%%%%%%%%%%%%%%%%%%%%%%%%%%%%%%%%%%%%%%%%%%%%%%
\usepackage{titlesec}
\usepackage{sectsty}
%%%%%%%%%%%%%%%%%%%%%%%%
%set section/subsections HEADINGS font and color
\sectionfont{\color{MyColor1}}  % sets colour of sections
\subsectionfont{\color{MyColor1}}  % sets colour of sections

%set section enumerator to arabic number (see footnotes markings alternatives)
\renewcommand\thesection{\arabic{section}.} %define sections numbering
\renewcommand\thesubsection{\thesection\arabic{subsection}} %subsec.num.

%define new section style
\newcommand{\mysection}{
\titleformat{\section} [runin] {\usefont{OT1}{lmss}{b}{n}\color{MyColor1}} 
{\thesection} {3pt} {} } 

%%%%%%%%%%%%%%%%%%%%%%%%%%%%%%%%%%%%%%%%%%%%%%%%%%
%		CAPTIONS
%%%%%%%%%%%%%%%%%%%%%%%%%%%%%%%%%%%%%%%%%%%%%%%%%%
\usepackage{caption}
\usepackage{subcaption}
\usepackage{hyperref}
\usepackage[export]{adjustbox}
%%%%%%%%%%%%%%%%%%%%%%%%
\captionsetup[figure]{labelfont={color=Turquoise}}

%%%%%%%%%%%%%%%%%%%%%%%%%%%%%%%%%%%%%%%%%%%%%%%%%%
%		!!!EQUATION (ARRAY) --> USING ALIGN INSTEAD
%%%%%%%%%%%%%%%%%%%%%%%%%%%%%%%%%%%%%%%%%%%%%%%%%%
%using amsmath package to redefine eq. numeration (1.1, 1.2, ...) 
%%%%%%%%%%%%%%%%%%%%%%%%
\renewcommand{\theequation}{\thesection\arabic{equation}}

%set box background to grey in align environment 
\usepackage{etoolbox}% http://ctan.org/pkg/etoolbox
\makeatletter
\patchcmd{\@Aboxed}{\boxed{#1#2}}{\colorbox{black!15}{$#1#2$}}{}{}%
\patchcmd{\@boxed}{\boxed{#1#2}}{\colorbox{black!15}{$#1#2$}}{}{}%
\makeatother
%%%%%%%%%%%%%%%%%%%%%%%%%%%%%%%%%%%%%%%%%%%%%%%%%%




%%%%%%%%%%%%%%%%%%%%%%%%%%%%%%%%%%%%%%%%%%%%%%%%%%
%% DESIGN CIRCUITS
%%%%%%%%%%%%%%%%%%%%%%%%%%%%%%%%%%%%%%%%%%%%%%%%%%
\usepackage[siunitx, american, smartlabels, cute inductors, europeanvoltages]{circuitikz}
%%%%%%%%%%%%%%%%%%%%%%%%%%%%%%%%%%%%%%%%%%%%%%%%%%



\makeatletter
\let\reftagform@=\tagform@
\def\tagform@#1{\maketag@@@{(\ignorespaces\textcolor{red}{#1}\unskip\@@italiccorr)}}
\renewcommand{\eqref}[1]{\textup{\reftagform@{\ref{#1}}}}
\makeatother
\usepackage{hyperref}
\usepackage{float}
\hypersetup{colorlinks=false}

%%%%%%%%%%%%%%%%%%%%%%%%%%%%%%%%%%%%%%%%%%%%%%%%%%
%% PREPARE TITLE
%%%%%%%%%%%%%%%%%%%%%%%%%%%%%%%%%%%%%%%%%%%%%%%%%%
\title{\blue CS$425$: Computer Networks \\
\blueb Project-$5$: Internet Measurements}
\author{Shubham Agrawal \\
13674, \href{mailto:agshubh191@gmail.com}{agshubh191@gmail.com}}
\date{\today}
%%%%%%%%xgvim%%%%%%%%%%%%%%%%%%%%%%%%%%%%%%%%%%%%%%%%%%



\begin{document}
\maketitle
\section{Traffic Measurement}
\subsection{Average Packet Size in Trace}
Average Packet size : 405.860461949. For each record i in table, I divided total bytes transferred (records[i][“doctets”]) by total number of packets (records[i][“dpkts”]). This will give me average flow for that record as avgFlow[i]. I computed avgFlow for every record i and took average. 

\subsection{CCDF Plots of flow durations and flow sizes}
\begin{figure}[H]
\centering
\begin{subfigure}{.5\textwidth}
  \centering
  \includegraphics[width=\linewidth]{CCDF-Duration-Linear.png}
  \caption{Linear Scale}
  \label{fig:sub1}
\end{subfigure}%
\begin{subfigure}{.5\textwidth}
  \centering
  \includegraphics[width=\linewidth]{CCDF-Duration-Log.png}
  \caption{Log scale on both axes}
  \label{fig:sub2}
\end{subfigure}
\caption{CCDF of Duration (in seconds) vs Number of records}
\label{fig:test}
\end{figure}

\begin{figure}[H]
\centering
\begin{subfigure}{.5\textwidth}
  \centering
  \includegraphics[width=\linewidth]{CCDF-Flow-Size-Bytes-Linear.png}
  \caption{Linear Scale}
  \label{fig:sub1}
\end{subfigure}%
\begin{subfigure}{.5\textwidth}
  \centering
  \includegraphics[width=\linewidth]{CCDF-Flow-Size-Bytes-Log.png}
  \caption{Log scale on both axes}
  \label{fig:sub2}
\end{subfigure}
\caption{CCDF-Flow of flow-Size in terms of number of bytes transferred}
\label{fig:test}
\end{figure}

\begin{figure}[H]
\centering
\begin{subfigure}{.5\textwidth}
  \centering
  \includegraphics[width=\linewidth]{CCDF-Flow-Size-Packets-Linear.png}
  \caption{Linear Scale}
  \label{fig:sub1}
\end{subfigure}%
\begin{subfigure}{.5\textwidth}
  \centering
  \includegraphics[width=\linewidth]{CCDF-Flow-Size-Packets-Log.png}
  \caption{Log scale on both axes}
  \label{fig:sub2}
\end{subfigure}
\caption{CCDF of Duration (in seconds) vs Number of records}
\label{fig:test}
\end{figure}
What are the main features of
the graphs? 
What artifacts of Netflow and of network protocols could be responsible for
these features?
Why is it useful to plot on a logarithmic scale?
32897-33059	unassigned

\subsection{Common Port Numbers}

\begin{table}[H]
	\centering
	\begin{tabular}{ |c|c|l| } 
		\hline
		\textbf{Port} & \textbf{Traffic percentage} & \textbf{Description} \\ \hline
		80      & 43.939    & Hypertext Transfer Protocol (HTTP) \\ \hline
		33001   & 7.3627    & Unassigned \\ \hline 
		1935    & 3.6642    & Macromedia Flash Communications Server MX\\ \hline
		22      & 2.1682    & SSH \\ \hline
		443     & 1.7256    & Hypertext Transfer Protocol over TLS/SSL (HTTPS)\\ \hline
		55000   & 1.6235    & \\ \hline
		388     & 1.3387    & \\ \hline
		16402   & 0.7621    & Real-time Transport Protocol (RTP)\\ \hline
		20      & 0.6717    & File Transfer Protocol (FTP) data transfer\\ \hline
		0       & 0.6354    & Reserved \\ \hline
	\end{tabular}
	\caption{Top ten source IP ports, volume of traffic and description}
\end{table}



\begin{table}[H]
	\centering
	\begin{tabular}{ |c|c|l| } 
		 \hline
		\textbf{Port} & \textbf{Traffic percentage} & \textbf{Description} \\ \hline
		33002   & 4.0248    & \\ \hline
		80      & 2.9274    & Hypertext Transfer Protocol (HTTP) \\ \hline
		49385   & 2.0916    & \\ \hline
		62269   & 1.2293    & \\ \hline
		443     & 0.7741    & Hypertext Transfer Protocol over TLS/SSL (HTTPS)\\ \hline
		43132   & 0.7630    & \\ \hline
		16402   & 0.7428    & Real-time Transport Protocol (RTP)\\ \hline
		22      & 0.6483    & SSH \\ \hline
		5500    & 0.6477    & VNC remote desktop protocol \\ \hline
		0       & 0.6108    & Reserved \\ \hline
	\end{tabular}
	\caption{Top ten destination IP ports, volume of traffic and description}	
\end{table}

Explain any significant differences between the results for sender vs. receiver port numbers ? 

\subsection{Traffic volumes based on the source IP prefix}


\begin{table}[H]
	\centering
	\begin{tabular}{ |c|c|c|c| } 
		\hline
& Top 0.1 \%& Top 1 \%& Top 10\% \\ \hline
Ip prefixes with 0 length mask not included & 0.5698 & 0.7910 & 0.9721 \\ \hline
Ip prefixes with 0 length mask included & 0.2831 & 0.6377 & 0.9511 \\ \hline 
	\end{tabular}
	\caption{ Traffic volume based on source ip prefix (with and without ip prefixes with 0 length mask) }	
\end{table}


\subsection{Traffic sent to/from Princeton}
Address block for Princeton is 128.112.0.0/16 . What fraction of the traffic (by bytes and by packets) in the trace is sent by Princeton? To Princeton


\begin{table}[H]
	\centering
	\begin{tabular}{ |c|c|c| } 
		\hline
Traffic fraction & By bytes & By packets \\ \hline
To Princeton & 0.02163 & 0.01380 \\ \hline
From Princeton & 0.00700 & 0.01014 \\ \hline
	\end{tabular}
	\caption{ Traffic sent to/from Princeton in terms of bytes and packets transferred}	
\end{table}


\section{BGP Measurement}
\subsection{BGP updates per minute}
\begin{table}[H]
	\centering
	\begin{tabular}{ |c|c|c|c|}
		\hline
		Session & Session 1 & Session 2 & Session 3\\ \hline
		BGP Updates per minute & 657.5166666666667& 1068.1 & 3165.016666666667 \\ \hline
	\end{tabular}
\end{table}
\textbf{Describe how you computed this result}

\subsection{Fraction of IP prefixes that experience no update messages}
\begin{table}[H]
	\centering
	\begin{tabular}{ |c|c|c|c|}
		\hline
 Session & Session 1 & Session 2 & Session 3\\ \hline
 Prefixes with no update &0.9890216060208861 &0.9649797854760322 &0.9068765372573193 \\ \hline
	\end{tabular}
\end{table}

\subsection{What prefix (or prefixes) experiences the most updates, and how frequent are they}
\begin{table}[H]
	\centering
	\begin{tabular}{ |c|c|c|}
		\hline
 Session 1 & 121.52.149.0/24 & 0.012927428962510456 \\ \hline
 &121.52.144.0/24 & 0.012927  \\ \hline
 & 121.52.145.0/24 & 0.012927  \\ \hline
 &121.52.150.0/24 &  0.012927  \\ \hline
 Session 2 &85.239.28.0/24 &  0.00573  \\ \hline
 Session 3 &109.161.64.0/20 &  0.00191  \\ \hline
	\end{tabular}
\end{table}

\subsection{Fraction of all update messages that comes from the most unstable prefixes}
\begin{table}[H]
	\centering
	\begin{tabular}{ |c|c|c|c|}
		\hline
Sessions & Top 0.1\% & Top 1\% & Top 10\%\\ \hline
Session 1 & 0.05881 & 0.17905& 0.40104 \\ \hline
Session 2 & 0.06318& 0.17979&  0.50111 \\ \hline
Session 3 & 0.02891 & 0.069725 & 0.22629 \\ \hline
	\end{tabular}
\end{table}



\section{Summary}
\begin{itemize}
\item All the mentioned features seems to work
\item All the test-cases mentioned pass with changing "read$\_$all" to "read$\_$some" as discussed on slack
\item Sometimes test-cases does not pass due to slow internet connection 
\end{itemize}

\section{Appendix}
\subsection{Code \textit{proxy.c}}
 
\begin{lstlisting}
#include "proxy_parse.h"
#include<iostream>
#include<stdio.h>
#include<sys/socket.h>
#include<netinet/in.h>
#include<arpa/inet.h>
#include<unistd.h>
#include<string.h>
#include<iostream>
#include<string>
#include<csignal>
#include<netdb.h>
#include<unistd.h>
#include<sys/types.h>
using namespace std;

#define maxClientsWaiting 100  // Max number of waiting clients
#define bufferSize 9000 
#define smallBufferSize 1000 
#define maxForkedChilds 20
char proxyRequestFormat[] = "GET %s HTTP/1.0\r\n%s";
char defaultProxyPort[] = "80\0";
int numForkedChild = 0;
char errResponseFormat[] = "HTTP/1.0 %d %s\r\n\r\n";

struct status{
	int status;
	char msg[1000];
};
typedef struct status status;

status S500 = {500, "Internal Error\0"};

// To get html for response error
void sendError(int sockid, status resStatus){
	char buffer[smallBufferSize];
	bzero(buffer, smallBufferSize);
	snprintf(buffer, smallBufferSize, errResponseFormat, resStatus.status, resStatus.msg);
	send(sockid, buffer, strlen(buffer),0);
	return;
}

//To update number of forked children
void updateNumForkedChild(int update){
	//TODO: Make it shared and race free in parent and child
	numForkedChild = numForkedChild + update;
	return;
}

int hostname_to_ip(char * hostname , char* ip)
{
    struct hostent *he;
    struct in_addr **addr_list;
    int i;
         
    if ( (he = gethostbyname( hostname ) ) == NULL) 
    {
        // get the host info
        //herror("gethostbyname");
		//printf("get host by name error");
        return 1;
    }
 
    addr_list = (struct in_addr **) he->h_addr_list;
     
    for(i = 0; addr_list[i] != NULL; i++) 
    {
        //Return the first one;
        strcpy(ip , inet_ntoa(*addr_list[i]) );
        return 0;
    }
    //printf("%d\n", i) ;
    return 1;
}

int getProxyRequest(ParsedRequest* req, char* buffer){
	//TODO: Fill headers
	
	// Set a specific header (key) to a value. In this case,
	//we set the "Last-Modified" key to be set to have as 
	//value  a date in February 2014 

	if(ParsedHeader_set(req, "Connection", "close") < 0){
		//printf("Set header key not work\n");
		return -1;
	}
	
	if(ParsedHeader_set(req, "Host", req->host) < 0){
		//printf("Set header key not work\n");
		return -1;
	}
	// Turn the headers from the request into a string.
	int rlen = ParsedHeader_headersLen(req);
	char buf[rlen+1];
	if (ParsedRequest_unparse_headers(req, buf, rlen) < 0) {
		//printf("unparse failed\n");
		return -1;
	}
	buf[rlen] ='\0';
	//printf("Header String........\n%s\n.......\n", buf);
	//print out buf for text headers only 
	
	snprintf(buffer, bufferSize, proxyRequestFormat, req->path, buf);
	return 0;
}

int main(int argc, char* argv[]){
 	if(argc<2){
		//printf("Usage ./proxy portNumber");
		return 0;
	}
	int port = atoi(argv[1]);
	int sockid = socket(AF_INET,SOCK_STREAM, 0); 
	if(sockid<0){
		//printf("Unable to get sockid");
		return 0;
	}
	struct sockaddr_in serverAdd;
	serverAdd.sin_family = AF_INET;
	serverAdd.sin_port=htons(port);
	serverAdd.sin_addr.s_addr=INADDR_ANY;
	int bind_status = bind(sockid, (struct sockaddr*)&serverAdd, sizeof(serverAdd));
	if(bind_status==0){
		//printf("bind successful\n");
	}
	else{
		//printf("bind failed\n");
		return 0; 
	}
	int listen_status = listen(sockid, maxClientsWaiting);
	if(listen_status==0){
		//printf("Listening ...\n");
	}
	else{
		//printf("Listening failed\n");
		return 0;
	}
	// accept connections in loop

	while(1){
		struct sockaddr clientAdd;
		socklen_t addrLen;
		int newSockid = accept(sockid, &clientAdd, &addrLen);
		// Keep waiting until
		for (; numForkedChild >= maxForkedChilds; --numForkedChild){
			//printf("Waiting for some(%d) children to exit\n", numForkedChild-maxForkedChilds);
			wait();
		}
		if(fork()==0){
			close(sockid); //Child will not listen
			char buffer[bufferSize];
			char bufferTemp[bufferSize];
			bzero(buffer, strlen(buffer));
			int n;
			do{
				bzero(bufferTemp, strlen(bufferTemp));
				n = recv(newSockid, bufferTemp, bufferSize, 0);
				strcat(buffer, bufferTemp);
			}while((strstr(buffer, "\r\n\r\n")==NULL));
			//printf("--------Request Recieved(%d)------\n%s\n-----------------\n", n, buffer);

			//Parse request 
			int len = strlen(buffer); 
			//Create a ParsedRequest to use. This ParsedRequest
			//is dynamically allocated.
			ParsedRequest *req = ParsedRequest_create();
			if (ParsedRequest_parse(req, buffer, len) < 0) {
				//printf("Request parsing failed\n");
				sendError(newSockid, S500);
				return 0;
			}
			//printf("Request parsing completed\n");

			if(strcmp(req->method, "GET") != 0 ){
				//printf("Method %s not implemented\n", req->method);
				sendError(newSockid, S500);
				return 0;
			}
			if(req->port==NULL){
				req->port = (char*)malloc(6);
				bzero(req->port, 6);
				strcpy(req->port, defaultProxyPort);
			}

			char ip[50];
			bzero(ip, 50);
			hostname_to_ip(req->host, ip);
			int fSockid = socket(AF_INET,SOCK_STREAM, 0); 
			struct sockaddr_in fServerAdd;
			fServerAdd.sin_family = AF_INET;
			fServerAdd.sin_port=htons(atoi(req->port));
			fServerAdd.sin_addr.s_addr=inet_addr(ip);
			int fAddrLen = sizeof(fServerAdd);
			int conn_status = connect(fSockid,(struct sockaddr*)&fServerAdd, fAddrLen);
			if(conn_status==0){
				//printf("Connection to host established\n");
			}
			else{
				//printf("Connection to host failed: %d\n ", conn_status);
				sendError(newSockid, S500);
				return 0;
			}
			char proxyRequest[bufferSize];	
			bzero(proxyRequest, bufferSize);
			if(getProxyRequest(req, proxyRequest)!= 0){
				//printf("Proxy request unable to create\n");
				sendError(newSockid, S500);
				return 0;
			}
			//printf("--------Proxy sending request------\n%s\n-----------------\n", proxyRequest);
			n = send(fSockid,proxyRequest,strlen(proxyRequest),0);
			bzero(buffer, strlen(buffer));
			int bytes = 0;
			//printf("--------Response from host recieved------\n");
			while((n = recv(fSockid, buffer, bufferSize, 0)) != 0){
				//printf("%s",buffer);
				// Send to client
				send(newSockid,buffer,strlen(buffer),0);
				bytes = bytes+n;
				bzero(buffer, strlen(buffer));
			}
			//printf("\n-----------------\n");
			close(newSockid);
			return 0;
		}
		else{
			updateNumForkedChild(1);
			close(newSockid);
		}
	}
	return 0;
}

\end{lstlisting}
\end{document}